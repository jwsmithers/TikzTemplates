\documentclass[10pt,border=10pt,tikz]{standalone}
\usepackage{tikz,tkz-euclide,pdftexcmds}
\usetikzlibrary{backgrounds}
\pgfdeclarelayer{foreground}
\pgfsetlayers{background,main,foreground}
\usetkzobj{all}
\usepackage{amsmath}
\usepackage{graphicx}            
\usepackage{amssymb}
\usetikzlibrary{shapes,snakes}

\makeatletter

\pgfdeclareradialshading[tikz@ball]{my ball}{\pgfqpoint{5bp}{10bp}}{%
 color(0bp)=(tikz@ball!10!white);
 color(8bp)=(tikz@ball!65!white);
 color(12bp)=(tikz@ball);
 color(25bp)=(tikz@ball!70!black);
 color(50bp)=(black)}

\pgfdeclareradialshading[tikz@ball]{new ball}{\pgfqpoint{12bp}{12bp}}{%
 color(0cm)=(tikz@ball!10!white);
 color(0.5cm)=(tikz@ball!50!white);
 color(0.75cm)=(tikz@ball);
 color(0.9cm)=(tikz@ball!80!black);
 color(1cm)=(tikz@ball!60!black)} 

\pgfdeclareradialshading{ball shadow}{\pgfpointorigin}%
{color(0cm)=(black);
color(2mm)=(gray!100);
color(3mm)=(gray!70);
color(7mm)=(white)
} 

 % to make possible use "myball color=..." 
\tikzoption{my ball color}{\pgfutil@colorlet{tikz@ball}{#1}\def\tikz@shading{my ball}\tikz@addmode{\tikz@mode@shadetrue}}
\tikzoption{new ball color}{\pgfutil@colorlet{tikz@ball}{#1}\def\tikz@shading{new ball}\tikz@addmode{\tikz@mode@shadetrue}}
\tikzoption{ball shadow color}{\pgfutil@colorlet{tikz@ball}{#1}\def\tikz@shading{ball shadow}\tikz@addmode{\tikz@mode@shadetrue}}

% Can't remember where I got this function otherwise I'd link to it
\newcommand{\ball}[8][white]{%
        \begin{scope}[shift = {(#2,#3)}]
        \begin{scope}[on background layer]
        		  \node[shading=ball shadow,xscale=2,yscale=0.3,circle,minimum size=7mm] 
            	  at (0.15,-0.225){};
         \end{scope}
          % Special shading for white
          \ifnum\pdf@strcmp{#1}{white}=\z@%  
            \shade[my ball color=#1] (0.15,0.25) circle (0.5);
          \else
            \shade[new ball color=#1] (0.15,0.25) circle (0.5);
          \fi
        \end{scope}
        
        \draw (#2+0.15,#3+0.2) node[fill=none] (#8) {\textbf{\Large{#4}}};  % flavour
	\draw (#2+0.15,#3+0.55) node[fill=none] {\scalebox{0.4}{#5}};	% mass
	\draw (#2-0.15,#3+0.2) node[fill=none] {\scalebox{0.4}{#6}};	% charge
	\draw (#2+0.48,#3+0.20) node[fill=none] {\scalebox{0.4}{#7}};	% spin
}

\makeatother

\begin{document}

\begin{tikzpicture}     
% Declare 2 boxes since i'm lazy and don't wont to look up how to provide an argument
    \tikzset{
        smallboundingbox/.style={rectangle,rounded corners,draw=black, fill=none, thick, inner sep=1em,minimum width=6.5em,minimum height=3.3em, text centered},
        myarrow/.style={->, >=latex', shorten >=1pt, thick},
        mylabel/.style={text width=7em, text centered} 
    }  
    \tikzset{
        boundingbox/.style={rectangle,rounded corners,draw=black, fill=none, thick, inner sep=1em,minimum width=14em,minimum height=5.9em, text centered},
        myarrow/.style={->, >=latex', shorten >=1pt, thick},
        mylabel/.style={text width=7em, text centered} 
    }  

% labels:
    \node[draw,draw=none,text=gray, rotate=90] at (-0.7,-0.2) {Quarks};
    \node[draw,draw=none,text=gray, rotate=90] at (-0.7,-2) {Mediators};
    \node[draw,draw=none,text=gray, rotate=90] at (-0.7,-4) {Leptons};
    \node[draw,draw=none, text=gray] at (1,1) {I};
    \node[draw,draw=none,text=gray] at (2.5,1) {II};
    \node[draw,draw=none,text=gray] at (4,1) {III};

    % Example:
    %\ball[color]{x}{y}{label}{mass}{charge}{spin}{labelforpath}%u

% Quarks
    \ball[orange]{0.5}{0}{$u$}{$\sim2.2 \, \mathrm{MeV}$}{$2/3$}{$1/2$}{u}%u
    \ball[orange]{1}{-0.9}{$d$}{$\sim4.7 \, \mathrm{MeV}$}{$-1/3$}{$1/2$}{d} %d

    \ball[orange]{2}{0}{$c$}{$\sim1.3 \, \mathrm{GeV}$}{$2/3$}{$1/2$}{c} %c
    \ball[orange]{2.5}{-0.9}{$s$}{$\sim100 \, \mathrm{MeV}$}{$-1/3$}{$1/2$}{s}  %s

    \ball[orange]{3.5}{0}{$t$}{$173 \, \mathrm{GeV}$}{$2/3$}{$1/2$}{t} %t
    \ball[orange]{4}{-0.9}{$b$}{$\sim4.2 \, \mathrm{GeV}$}{$-1/3$}{$1/2$}{b}  %b
   
     \node[boundingbox] at (2.4,-0.2) (quarkbox)  {};
     
% gauge bosons
    \ball[white]{0.5}{-2.3}{$g$}{$0$}{$0$}{$1$}{g} %g
    \ball[white]{1.8}{-2.3}{\small{$Z$}}{$91.2 \, \mathrm{GeV}$}{$0$}{$1$}{Z} %Z
    \ball[white]{2.9}{-2.3}{\small{$W$}}{$80.4 \, \mathrm{GeV}$}{$\pm1$}{$1$}{W}  %W
 
    \node[smallboundingbox] at (2.5,-2.05) (bosonbox)  {};
    
    \ball[white]{4.6}{-2.3}{{\small $H$}}{$125 \, \mathrm{GeV}$}{$0$}{$0$}{H} %H
    \ball[white]{5.8}{-2.3}{$\gamma$}{$0$}{$0$}{$1$}{gamma}  %gamma


% Leptons    
    \ball[purple!60]{0.5}{-3.7}{$\nu_e$}{$\gtrsim0$}{$0$}{$1/2$}{nue}  %nu_e
    \ball[purple!60]{1}{-4.6}{$e$}{$0.5 \, \mathrm{MeV}$}{$-1$}{$1/2$}{e} %e
    
    \ball[purple!60]{2}{-3.7}{$\nu_{\mu}$}{$\gtrsim0$}{$0$}{$1/2$}{numu}  %nu_mu
    \ball[purple!60]{2.5}{-4.6}{$\mu$}{$105 \, \mathrm{MeV}$}{$-1$}{$1/2$}{mu} %mu

    \ball[purple!60]{3.5}{-3.7}{$\nu_{\tau}$}{$\gtrsim0$}{$0$}{$1/2$}{nutau} %nu_tau
    \ball[purple!60]{4}{-4.6}{$\tau$}{$1.78 \, \mathrm{GeV}$}{$-1$}{$1/2$}{tau} %tau

    \node[boundingbox] at (2.4,-3.9) (leptonbox)  {};

    \node[draw,draw=none, text=gray] at (1,-5.1) {I};
    \node[draw,draw=none,text=gray] at (2.5,-5.1) {II};
    \node[draw,draw=none,text=gray] at (4,-5.1) {III};


% legend
    
    \ball[blue!20]{5.8}{-0.5}{\small{$X$}}{\small{$mass$}}{\small{$charge$}}{\small{$spin$}}{legend}

% paths 
    \begin{scope}[on background layer]
    \draw (g) to[bend left=20] node {} (quarkbox);
   
% gamma to all
    \draw  (gamma) to[bend left] node {} (e);
    \draw  (gamma) to[bend left] node {} (mu);
    \draw  (gamma) to[bend left] node {} (tau);

    \draw  (gamma) to[bend right=10] node {} (quarkbox);

% Higgs to all
    \draw (H) to[bend left=60] node {} (e);
    \draw (H) to[bend left=50] node {} (mu);
    \draw (H) to[bend left=20] node {} (tau);
    
    \draw  (H) to[bend right] node {} (quarkbox);
    \draw  (H) to[bend left] node {} (bosonbox);

%W to photon
    \draw (W) to[bend left=40] node {} (gamma);
%Bosons to leptons
    \draw  (bosonbox) to[bend left=10] node {} (leptonbox);
    \draw  (bosonbox) to[bend left=10] node {} (quarkbox);

% Self couplings
    \draw (bosonbox) to [out=145,in=165,looseness=4] (bosonbox);
    \draw (H) to [out=150,in=180,looseness=15] (H);
    \draw  (g) to [out=150,in=180,looseness=20] (g);

    \end{scope}


\end{tikzpicture}
\end{document}



